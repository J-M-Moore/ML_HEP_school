\documentclass{article}

%%%% PAGE FORMATTING %%%%
\usepackage[margin=1in, includefoot]{geometry}
\usepackage{fancyhdr}
\pagestyle{fancy}
\fancyhead{}
\fancyfoot{}
\fancyfoot[R]{ \thepage\ }
\renewcommand{\footrulewidth}{1pt}
\renewcommand{\headrulewidth}{1pt}
\setlength{\parindent}{0pt}


%%%% FONTS %%%%
\usepackage[T1]{fontenc}
\usepackage[sfdefault]{AlegreyaSans} % sfdefault makes sans serif default.
\renewcommand*\oldstylenums[1]{{\AlegreyaSansOsF #1}}
\usepackage[usenames, dvipsnames]{color}


%%%% LISTS %%%%
\usepackage{enumitem}


%%%% TABLES %%%%
\usepackage{tabularx}
\usepackage{caption}
\captionsetup[table]{skip=10pt}
\newcolumntype{P}[1]{>{\centering\arraybackslash}p{#1}}
\setlength{\tabcolsep}{20pt}
\renewcommand{\arraystretch}{1}


%%%% FIGURES %%%%
\usepackage{graphicx}
\usepackage{float}


%%%% TIKZ PICTURES %%%%
\usepackage{tikz}
\usetikzlibrary{shapes}
\usetikzlibrary{positioning}
\usetikzlibrary{patterns}
\usepackage{pgf}
\usetikzlibrary{arrows.meta,bending}
\usetikzlibrary{decorations.pathmorphing}
\usetikzlibrary{decorations.markings}


%%%% MATHEMATICS %%%%
\usepackage{amsmath}
\usepackage{amsthm}
\usepackage{amssymb}
\usepackage{mathtools} % For prescripts.
\renewcommand{\vec}[1]{\textbf{#1}} % Vectors appear as boldface.
\usepackage{braket} % For braket notation.
\usepackage{tensor} % For tensor indices.


%%%% FRAMED MATHEMATICS ENVIRONMENTS %%%%
\usepackage{framed}

\newcounter{framedmath}[section]
\newenvironment{framethm}[1][]{\refstepcounter{framedmath} \begin{framed} \noindent \textbf{Theorem~\thesection.\theframedmath:} #1}{\end{framed}} 

\newenvironment{framethmstar}[1][]{\refstepcounter{framedmath} \begin{framed} \noindent \textbf{*Theorem~\thesection.\theframedmath:*} #1}{\end{framed}} 

\newenvironment{framelem}[1][]{\refstepcounter{framedmath} \begin{framed} \noindent \textbf{Lemma~\thesection.\theframedmath:} #1}{\end{framed}} 

\newenvironment{framecor}[1][]{\refstepcounter{framedmath} \begin{framed} \noindent \textbf{Corollary~\thesection.\theframedmath:} #1}{\end{framed}} 
   
\newenvironment{frameex}[1][]{\refstepcounter{framedmath} \begin{framed} \noindent \textbf{Example~\thesection.\theframedmath:} #1}{\end{framed}} 

\newenvironment{framedef}[1][]{\refstepcounter{framedmath} \begin{framed} \noindent \textbf{Definition~\thesection.\theframedmath:} #1}{\end{framed}} 

\newenvironment{frameax}[1][]{\refstepcounter{framedmath} \begin{framed} \noindent \textbf{Axiom~\thesection.\theframedmath:} #1}{\end{framed}} 

\newenvironment{frameprop}[1][]{\refstepcounter{framedmath} \begin{framed} \noindent \textbf{Proposition~\thesection.\theframedmath:} #1}{\end{framed}} 

%%%% MARGIN NOTES %%%%
\usepackage[fulladjust]{marginnote}

\makeatletter
\renewcommand\marginfont{\normalfont}
\renewcommand\raggedleftmarginnote{\@parboxrestore\@marginparreset\raggedleft}
\renewcommand\raggedrightmarginnote{\@parboxrestore\@marginparreset\raggedright}
\makeatother

\newcounter{lecture}
\newcommand{\lecture}{\refstepcounter{lecture}\reversemarginpar\marginnote{\ding{70} \textit{Lecture ~\thelecture}}}


%%%% FUN SYMBOLS %%%%
\usepackage{pifont}


%%%% PAGE RULES %%%%
\newcommand{\minirule}{\begin{center}\rule{0.7\textwidth}{.4pt}\end{center}}


%%%% CONTENTS LINES %%%%
\newcommand{\describesection}[1]{\addtocontents{toc}{\textit{\footnotesize \hspace{37pt} #1}\par\vspace{0.1cm}}}


%%%% HYPERLINKS %%%%
\usepackage{hyperref}
\hypersetup{
    colorlinks=true,
    linkcolor=red,
    filecolor=black,      
    urlcolor=red,
}


%%%% CODE %%%%
\usepackage{listings}
\usepackage{color}

\DeclareFixedFont{\ttb}{T1}{txtt}{bx}{n}{8} % for bold
\DeclareFixedFont{\ttm}{T1}{txtt}{m}{n}{8}  % for normal

\definecolor{deepblue}{rgb}{0,0,0.5}
\definecolor{deepred}{rgb}{0.6,0,0}
\definecolor{deepgreen}{rgb}{0,0.5,0}

\newcommand\pythonstyle{\lstset{
language=Python,
basicstyle=\ttm,
morekeywords={self},              % Add keywords here
keywordstyle=\ttb\color{deepblue},
emph={MyClass,__init__},          % Custom highlighting
emphstyle=\ttb\color{deepred},    % Custom highlighting style
stringstyle=\color{deepgreen},
frame=tb,                         % Any extra options here
showstringspaces=false,
numbers=left,
numberstyle=\small\color{gray}
}}

\lstnewenvironment{python}[1][]
{
\pythonstyle
\lstset{#1}
}
{}


%%%% FLAG ERROR IN NOTES %%%% 
\newcommand{\flag}[1]{{\color{red} #1}}
